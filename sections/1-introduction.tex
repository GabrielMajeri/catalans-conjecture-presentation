\section{Introduction}

\begin{frame}
\frametitle{Catalan's Conjecture}

The numbers \(2^3 = 8\) and \(3^2 = 9\) are the only consecutive non-zero integers which are perfect powers. \\[1em]

In other words, the equation
\[
    a^n - b^m = 1
\]
has no solutions over \(\integers^*\) except for \((\pm 3)^2 - 2^3 = 1\).
\end{frame}

\begin{frame}
\frametitle{Catalan's Conjecture}

\begin{figure}
    \centering
    \includegraphics[width=0.9\textwidth]{catalans-note}
    \caption*{Eugène Catalan's letter to the editors of Crelle's Journal \cite{Catalan1844_Note}}
\end{figure}
\end{frame}

\begin{frame}
\frametitle{Purpose of this talk?}

While the solution to a particular Diophantine equation in itself might be inconsequential, the methods and theories we develop along the way can have many applications. \\[1em]

The purpose of this presentation is to focus on the \textbf{concepts} and \textbf{results} involved in the proof of Catalan's conjecture, not necessarily on the calculatory details of the proofs (references will be given where appropriate).
\end{frame}

% \begin{frame}
% \frametitle{Reductions}

% If \(\abs{a} > 1\) and \(n < 0\), then \(\abs{x^n} = \frac{1}{x^{-n}} < 1\) and

% Analogously for \(\abs{b} > 1\) and \(m < 1\).
% \end{frame}

\begin{frame}
\frametitle{Reductions: composite exponents}

If \(a^n - b^m = 1\) is a solution to Catalan's equation and \(p\), \(q\) are two primes such that \(p \divides n\), \(q \divides m\) then
\[
    \left(a^{\frac{n}{p}}\right)^p - \left(b^{\frac{m}{q}}\right)^q = x^p - y^q = 1
\]
is another solution. Hence, it is enough to show that no solution exists for \textbf{prime} exponents.
\end{frame}

\begin{frame}
\frametitle{Reductions: equal odd exponents}

Suppose that \(x \neq 0\), \(y \neq 0\) are a solution to Catalan's equation for the \textbf{odd} exponent \(p = q\).
\pause
Then
\begin{gather*}
    x^p - y^p = 1 \\
    \iff (x - y)(x^{p - 1} + x^{p - 2} y + \dots + x y^{p - 2} + y^{p - 1}) = 1
\end{gather*}

\pause

Possibilities for the signs of \(x\) and \(y\) are:
{\scriptsize
\begin{itemize}
    \item If \(x \geq 1\), \(y \geq 1\) then \(x^{p - 1}, x^{p - 2} y, \dots, y^{p - 1}\) are all positive, so the second factor is \(\geq 2\).

    \item If \(x \leq -1\) and \(y \leq -1\), then \(x^{p - 1}, x^{p - 2} y, \dots, y^{p - 1}\) are all positive (since \(p - 1\) is \textbf{even}), so the second factor is \(\geq 2\).

    \item If \(x \geq 1\) and \(y \leq -1\), then \(x - y\) is \(\geq 2\).

    \item If \(x \leq -1\) and \(y \geq 1\), then \(x - y\) is \(\leq -2\).
\end{itemize}
}
In all scenarios, we get a contradiction. Hence, we may assume that \(p \neq q\) when \(p\) and \(q\) are both odd.
\end{frame}

\begin{frame}
\frametitle{Reductions: symmetry}

Suppose that \(x\), \(y\) are a solution to Catalan's equation for two \textbf{odd} primes \(p\), \(q\). Then
\[
    (-y)^q - (-x)^p = - y^q + x^p = x^p - y^q = 1
\]
Hence, the problem is symmetrical in \(p\) and \(q\).
\end{frame}
