\subsection{Wieferich primes}

\begin{frame}
\frametitle{Developments from Cassel's Relations}

Over the following years, the Finnish mathematicians K.\ Inkeri and S.\ Hyyrö managed to extend Cassel's relations and discover even more properties of Catalan's equation.
\end{frame}

\begin{frame}
\frametitle{Wieferich Primes}

To state their result, we will have to first define a new term:
\begin{definition}
Two primes \(p\) and \(q\) are called a \emph{Wieferich pair} if the following two conditions hold:
\begin{align*}
    p^{q - 1} &\equiv 1 \mod{q^2} \\
    q^{p - 1} &\equiv 1 \mod{p^2}
\end{align*}
\end{definition}
\end{frame}

\begin{frame}
\frametitle{Wieferich Primes}

Examples of Wieferich primes: \(2\) and \(1093\), \(3\) and \(1006003\), \(911\) and \(318917\), \(\dots\)

\begin{remark}
Wieferich pairs are \emph{extremly} rare. We know only 17 such pairs for \(p \cdot q < 10^{15}\).
\end{remark}
\end{frame}

\begin{frame}
\frametitle{Fractional ideals}

Let \(K\) be an algebraic number field and \(\symcal{O}_K\) its ring of integers.

\begin{definition}
A \emph{fractional ideal} of \(K\) is an \(\symcal{O}_K\)-submodule \(M\) of \(K\) for which there exists a non-zero \(\alpha \in \symcal{O}_K\) such that
\[
    \alpha M \subseteq \symcal{O}_K
\]
\end{definition}
\end{frame}

\begin{frame}
\frametitle{Fractional ideals}

\begin{definition}
Denote by \(\symrm{Fr}(K)\) the set of all fractional ideals of \(K\).
\end{definition}

\begin{theorem}
Every fractional ideal of \(K\) has a multiplicative inverse and \(\left(\symrm{Fr}(K), \cdot\right)\) is a group.
\end{theorem}
\end{frame}

\begin{frame}
\frametitle{Principal fractional ideals}
\begin{definition}
A fractional ideal is called \emph{principal} if it is generated by a single element.
\end{definition}

\begin{remark}
The principal fractional ideals \(\symrm{Pr}(K)\) form a subgroup of the group of fractional ideals.
\end{remark}
\end{frame}

\begin{frame}
\frametitle{Class group and class numbers}

\begin{definition}
The \emph{class group} of an algebraic number field \(K\) is the quotient of the group of fractional ideals by the subgroup of principal fractional ideals
\[
    \symrm{Cl}_{K} = \frac{\symrm{Fr}(K)}{\symrm{Pr}(K)}
\]
\end{definition}

\begin{definition}
The \emph{class number} of an algebraic number field \(K\) is the size of
\[
    h_K = \abs{Cl_K}
\]
\end{definition}

Intuitively, the class number can be used as a measure on how far a field's ring of integers is from a principal ideal domain.
\end{frame}

\begin{frame}
\frametitle{Wieferich Primes and Catalan's Equation}

Inkeri managed to link Wieferich primes to Catalan's conjecture. In 1964, he proved \cite{Inkeri1964} that:
\begin{theorem}[Inkeri, 1964]
Let \(\left(x, y, p, q\right)\) be a solution to Catalan's equation, with \(p\) and \(q\) prime numbers \(\geq 5\), \(p \equiv q \equiv 3 \mod{4}\), \(p > q\) and suppose that \textbf{\(q\) does not divide \(h\)}, the class number of the field \(\rationals\left(\sqrt{-p}\right)\). Then \(p\) and \(q\) are a double Wieferich pair, i.e.\
\[
    p^{q - 1} \equiv 1 \mod{q^2} \text{ and } q^{p - 1} \equiv 1 \mod{p^2}
\]
\end{theorem}
\end{frame}

\begin{frame}
\frametitle{Wieferich Primes and Catalan's Equation}
Over 25 years later, Inkeri generalized \cite{Inkeri1990} his result using the theory of cyclotomic fields:
\begin{theorem}[Inkeri, 1990]
Let \(\left(x, y, p, q\right)\) be a solution to Catalan's equation.
\begin{enumerate}
    \item If \(q \notdivides h_{\rationals\left(\zeta_p\right)}\) then \(x \equiv 0 \mod{q^2}\) and \(p^{q - 1} \cong 1 \mod{q^2}\)
    \item If \(p \notdivides h_{\rationals\left(\zeta_q\right)}\) then \(y \equiv 0 \mod{p^2}\) and \(q^{p - 1} \cong 1 \mod{p^2}\)
\end{enumerate}
\end{theorem}
\end{frame}

\begin{frame}
\frametitle{Inkeri's relations}

Inkeri's proofs start with Cassel's result, which says that
\[
    \frac{x^p - 1}{x - 1} = p u^q
\]
for some \(u \in \integers\), then factoring it in the field \(\rationals\left(\sqrt{-p}\right)\) or \(\rationals\left(\zeta_p\right)\) respectively.
\end{frame}

\begin{frame}
\frametitle{Inkeri's relations}

In his second paper, he uses the assumption
\[
    \gcd \left(q, h_{\rationals\left(\zeta_p\right)}\right) = 1
\]
in order to show that there exist \(\alpha, \beta \in \integers\left[\zeta_p\right]\) such that
\begin{align*}    
    \alpha^q + \overline{\alpha}^q = \varepsilon^p \\
    \beta^q + \overline{\beta}^q = \eta x
\end{align*}
where \(\varepsilon, \eta\) are \emph{\textbf{real} units} in \(\integers\left[\zeta_p\right]\). He then uses the factorisation of the ideal generated by \(q\) in \(\integers\left[\zeta_p\right]\) to reach his conclusion.
\end{frame}

\begin{frame}
\frametitle{Inkeri's relations}

Inkeri's theorem hints at the role cyclotomic fields are going to play in the proof of the general case. However, we will first have to get rid on the restrictive condition on the class number.
\end{frame}
