\subsection{Primary cyclotomic units}

\begin{frame}
\frametitle{Mihăilescu's proof of the general case}

We are now ready to discuss Mihăilescu's paper \cite{Mihailescu2004} from 2004, where he finally solved Catalan's conjecture in the general case.
\end{frame}

\begin{frame}
\frametitle{Cyclotomic units and real units}

An important automorphism of the field \(\rationals\left(\zeta_p\right)\) is the one given by complex conjugation, which we will denote by \(\sigma_{-1}\). \\[1em]

The field fixed by \(\sigma_{-1}\) is called the \emph{maximal real subfield} of \(\rationals\left(\zeta_p\right)\) and will be denoted by \(\rationals\left(\zeta_p + \overline{\zeta_p}\right)\) or \(\rationals\left(\zeta_p\right)^{+}\). Its ring of integers is \(\integers\left[\zeta_p\right]^{+} = \integers\left[\zeta_p + \overline{\zeta_p}\right]\).
\end{frame}

\begin{frame}
\frametitle{Cyclotomic units and real units}

We have the following situation:
\begin{align*}
    &\begin{tikzcd}
        \rationals\left(\zeta_p\right) \\
        \rationals\left(\zeta_p + \overline{\zeta_p}\right) \arrow[u, hookrightarrow] \\
        \rationals \arrow[u, hookrightarrow]
    \end{tikzcd}
    &
    &\begin{tikzcd}
        \set{\symrm{Id}} \arrow[d, hookrightarrow] \\
        G^{+} \arrow[d, hookrightarrow] \\
        G
    \end{tikzcd}
\end{align*}
\end{frame}

\begin{frame}
\frametitle{Cyclotomic units and real units}

\begin{definition}
Let \(E\) denote the multiplicative group of \emph{real units} of \(\rationals\left(\zeta_p\right)\), i.e.\ \(E = \reals \cap \left(\rationals\left(\zeta_p\right)^{\times}\right)\).
\end{definition}

\begin{definition}
Let \(C\) be the multiplicative group generated by elements of the form
\[
    \frac{\zeta_p^k - \overline{\zeta_p^k}}{\zeta_p - \overline{\zeta_p}} \in \reals
\]
which are called \emph{cyclotomic units}.
\end{definition}
\end{frame}

\begin{frame}
\frametitle{Cyclotomic units and real units}

\begin{proposition}
\(C\) is a subgroup of \(E\) of finite index.
\end{proposition}

\vspace{1em}

This result goes back to Kummer, and it was proven using methods from analytic number theory (characters and \(L\)-functions). \\[1em]

In fact, Kummer proved an even more precise statement:
\[
    \left[E : C\right] = h_{\rationals\left(\zeta_p + \overline{\zeta_p}\right)}
\]
\end{frame}

\begin{frame}
\frametitle{Mihăilescu's proof of the general case}

\begin{definition}
An algebraic integer \(\alpha \in \integers\left[\zeta_p\right]\) is called \emph{\(q\)-primary} if there exist another \(\beta \in \integers\left[\zeta_p\right]\) such that
\[
    \alpha \equiv \beta^q \mod{(q^2) \, \integers\left[\zeta_p\right]}
\]
\end{definition}

\begin{definition}
A cyclotomic unit which is \(q\)-primary will be called a \emph{\(q\)-primary unit}. Denote the set of \(q\)-primary units by \(C_q\).
\end{definition} 

\vspace{1em}

Mihăilescu's overall proof strategy will be to focus on the relation between the subgroup \(C_q\) and \(C\).
\end{frame}

\begin{frame}
\frametitle{Mihăilescu's proof of the general case}

Mihăilescu showed that not all cyclotomic units can be \(q\)-primary if \(q \leq p\):
\begin{proposition}
If \(C_q = C\), then \(q > p\).
\end{proposition}

\vspace{1em}

Mihăilescu proved this by considering the polynomial
\[
    f(X) = \frac{(1 + X)^q - (1 + X^q)}{q X} \mod{q} \in \finitefield{q}[X]
\]
and computing how many zeros it can have in the field \(\integers\left[\zeta_p\right] / \symfrak{q}\), where \(\symfrak{q}\) is a prime ideal sitting above \(q\).
\end{frame}

\begin{frame}
\frametitle{Thaine's Theorem}

We are going to need the following result:
\begin{theorem}[Thaine, 1988]
If \(p \not\equiv 1 \mod{q}\) and \(\theta \in \integers\left[G^{+}\right]\) annihilates the \(q\)-Sylow subgroup of \(E / C\), then \(\theta\) annihilates the \(q\)-Sylow subgroup of the class group of \(\rationals\left(\zeta_p + \overline{\zeta_p}\right)\).
\end{theorem}

\vspace{1em}

Thaine's original proof \cite{Thaine1988} appeared 1988 in the Annals of Mathematics, and relies on several deep theorems from class field theory, including Chebotarev's density theorem.
\end{frame}

\begin{frame}
\frametitle{\(q\)-primary units and Catalan's equation}

Using Thaine's theorem, Mihăilescu proved the following:
\begin{figure}
    \centering
    \includegraphics[width=0.9\textwidth]{bumira-camar}
\end{figure}
where \(C^q\) is the set of \(q\)-th powers of cyclotomic units and \(\Omega \subseteq \integers[G]\) is the smallest submodule which annihilates \(C_q / C^q\). In particular, Mihăilescu proved that if \(q \notdivides p - 1\) and \(\operatorname{ann}(C_q / C^q) = \Set{ 0 }\), then \(C = C_q\).
\end{frame}

\begin{frame}
\frametitle{Bumira Camar}

Why the name for the theorem?

\begin{figure}
    \centering
    \includegraphics[width=0.9\textwidth]{bumira-camar-remark}
\end{figure}
\end{frame}

\begin{frame}
\frametitle{\(q\)-primary units and Catalan's equation}

Mihăilescu uses this theorem to link \(\Omega\) to solutions of Catalan's equation. He showed that a nontrivial solution would imply \(\Omega = \symrm{N}\) (the ideal generated by the norm element) and that all cyclotomic units would be \(q\)-primary.
\end{frame}

\begin{frame}
\frametitle{Mihăilescu's proof of Catalan's conjecture}

The final part of the proof:
\begin{theorem}
Let \(\left(x, y, p, q\right)\) be a solution to Catalan's equation for \(p\), \(q\) odd primes and suppose that \(p \not\equiv 1 \mod{q^2}\). Then \(C = C_q\).
\end{theorem}

\vspace{1em}

This theorem is proven by writing the formal power series for the ``elementary q-th root'' of an algebraic number and doing a few estimates for what its value can be, then connecting this with the ideal \(\Omega\) from before.
\end{frame}

\begin{frame}
\frametitle{What if \(p \equiv 1 \mod{q^2}\)?}

Notice that in the previous theorem we made the assumption that \(p \not\equiv 1 \mod{q^2}\) or, in other words, that \(q^2\) does \emph{not} divide \(p - 1\). \\[1em]

Mihăilescu had to manually rule out this case separately, by using estimates similar to the ones for Tijdeman's bound. Later, others gave algebraic proofs for this result.
\end{frame}

\begin{frame}
\frametitle{Mihăilescu's proof of Catalan's conjecture}

\begin{theorem}
Catalan's conjecture is true.
\end{theorem}
\begin{proof}
Suppose that \((x, y, p, q)\) is a solution to Catalan's equation with \(p\) and \(q\) odd primes and \(p \not\equiv 1 \mod{q}\). Then \(C = C_q\) by using Mihăilescu's results, but an earlier proposition shows that this must imply \(q > p\), a contradiction.
\end{proof}

\end{frame}
