\section{Ko Chao's proof of the case \texorpdfstring{\(p = 2\)}{p = 3}}

\begin{frame}
\frametitle{The case \texorpdfstring{\(p = 2\)}{p = 2}}

\begin{theorem}[Ko Chao, 1965]
The equation
\[
    x^2 - y^q = 1
\]
has no solutions over \(\integers\) for \(q \geq 5\).
\end{theorem}
\end{frame}

\begin{frame}
\frametitle{The case \texorpdfstring{\(p = 2\)}{p = 2}}

\begin{remark}
We do not have to check the cases \(q = 2\) or \(q = 3\), since we've already ruled those out earlier.
\end{remark}
\end{frame}

\begin{frame}
\frametitle{Ko Chao's proof of the case \texorpdfstring{\(p = 2\)}{p = 2}}

Ko Chao's proof was originally published in 1960 in the journal of the University of Sichuan. It became better-known after it was republished in 1965 in the journal \emph{Acta Sinica}. Since then, several versions and simplifications of the proof have been produced. \\[1em]

The proof relies only on elementary computations and modular arithmetic. A simplified version of it can be found in E. Z. Chein's paper \cite{Chein1976}.
\end{frame}
